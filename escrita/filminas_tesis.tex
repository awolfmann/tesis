\documentclass[12pt,center]{beamer}

\input{tex/pre.tex}
%%%%%%%%%%% Lo que viene  a continuacion es el tipo de presentacion que quieres, 
% las presentaciones se llaman con nombres de ciudades puedes cambiarlas y tomar
%%%%%%%%la que mas te guste.
\mode<presentation> {
  %\usetheme{Frankfurt}
  %\usetheme{Warsaw}
  %\usetheme{Darmstadt}
   %\usetheme{Dresden}
  \usetheme{Singapore}
  %\usetheme{Bergen}
  %\usetheme{Boadilla}
  %\usetheme{BerKeley}
  \setbeamercovered{transparent}
  %\setbeamertemplate{background canvas}[vertical shading][bottom=red!20,top=yellow!30]
%   \setbeamertemplate{headline}{}
%%%%%%%%%%%%%%%%%%%%%%%%%%%%%%%%%%
%%%%%%% aqui vienen los colores

  %\usecolortheme{crane}
  %\usecolortheme{seahorse}
  \usecolortheme{whale}
  %\usecolortheme{rose}
  %\usecolortheme{orchid}
} \usepackage{alltt}



%%%%%%%los paquetes. 
\usepackage{amssymb,amsmath,latexsym}
%\usepackage[mathcal]{euscript}
%\usepackage[polish]{babel}
\usepackage{color}
\usepackage{hyperref}
%\usepackage{dsfont}
%\usepackage[normalem]{ulem}
\usepackage{enumerate}
%\usepackage[all,2cell,dvips]{xy} \UseAllTwocells \SilentMatrices
\usepackage[utf8]{inputenc}
\usepackage[spanish]{babel}
\usepackage{verbatim}
\usepackage{float}

\title{Transferencia de Estilo en Fotografias mediante Redes Neuronales Convolucionales}
%
%
\author{Ariel Wolfmann}

%
\institute{Facultad de Matemática, Astronomía, Física y Computación\\
	  Universidad Nacional de Córdoba}

\date{28 de Julio, 2017}


\begin{document}
%%%%%%%%%%%%%%%%%%%%%%%%%%%PAGINA DEL TITULO
\begin{frame}
  \titlepage
\end{frame}

%%%%%%%%%%%%%%%%%  tODO LO QUE QUIERAS PONER EN LOS FRAMES.
\begin{frame}
  \frametitle{Agenda}
  \tableofcontents[pausesections]
\end{frame}
  %  \tableofcontents[pausesections]
%   %You might wish to add the option [pausesections]


%%%%%%%%%%%%%%%%%%%%%%%% EFDs & Algebraic Functions %%%%%%%%%%%%%%%%%%%%%%%%%%%%
\section{Introducción}
  \begin{frame}{Contexto}

  \end{frame}	

  \begin{frame}{Motivación}

  \end{frame}	

\section{Aprendizaje Automatico}
  \begin{frame}{Definicion}
    Definicion
    Aprendizaje supervisado vs no supervisado: labels
  \end{frame}
	
  \begin{frame}{Clasificacion de imagenes}
    Enfoque clasico vs enfoque aprendizaje profundo
  \end{frame}

\section{Redes Neuronales Artificiales}
  \begin{frame}
    \frametitle{Funcion de perdida y optimizacion}
  \end{frame}

  \begin{frame}
      Retropropagacion
  \end{frame}


\section{Redes Neuronales Convolucionales}
\begin{frame}
    Funcion de convolucion
\end{frame}
\begin{frame}
\end{frame}

\section{Ajuste Fino}
\begin{frame}
 Ajuste Fino
\end{frame}


\section{Algoritmo de transferencia de estilo}
\begin{frame}
 \frametitle{}
 Funcion de perdida de estilo
\end{frame}
\begin{frame}
 Funcion de perdida del contenido
\end{frame}
\begin{frame}
 Proceso completo
\end{frame}

\begin{frame}
Hiperparametros
\end{frame}


\section{Elección automática de hiperparámetros}
\begin{frame}
 Descripcion del problema
\end{frame}
\begin{frame}
 Evaluacion
\end{frame}
\begin{frame}
  Solucion propuesta
\end{frame}
\begin{frame}
  Modulo de generacion de imagenes
\end{frame}
\begin{frame}
  Modulo de evaluacion de imagenes
\end{frame}

\section{Experimentos}
  \begin{frame}
    \frametitle{Reconocimiento de estilo}
  \end{frame}

  \begin{frame}
    \frametitle{Imagenes de ejemplo}
  \end{frame}

  \begin{frame}
    \frametitle{Eleccion del numero de iteraciones}
  \end{frame}
  
  \begin{frame}
    \frametitle{Graficos evaluacion vs numero de iteraciones}
  \end{frame}

\section{Conclusiones}
\begin{frame}{Conclusiones}
%%%%%%%
\end{frame}



\begin{frame}
 {\Huge ¿Preguntas?}
\end{frame}

\begin{frame}
 {\Huge ¡Gracias por escuchar!}
\end{frame}



% \begin{frame}%%%%%%%%%%
% \frametitle{Referencias}
% \begin{thebibliography}
% 
% %\pause
% \bibitem[KrCla79]{AE-classes}\textrm{Krauss, P.H., Clark, D.M.}, 
% \textit{Global Subdirect Products}, Amer. Math. Soc. Mem. \textbf{210} (1979).
% 
% 
% \end{thebibliography}
% 
% \end{frame}

\begin{frame}
\begin{figure}[center]
% \includegraphics[width=95mm]{19127354.jpg}
%\caption{algebraically closed clones in Post's Lattice} 
\end{figure}
\end{frame}

\end{document}
